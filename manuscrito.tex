\documentclass[11pt,]{article}
\usepackage[left=1in,top=1in,right=1in,bottom=1in]{geometry}
\newcommand*{\authorfont}{\fontfamily{phv}\selectfont}
\usepackage[]{mathpazo}


  \usepackage[T1]{fontenc}
  \usepackage[utf8]{inputenc}



\usepackage{abstract}
\renewcommand{\abstractname}{}    % clear the title
\renewcommand{\absnamepos}{empty} % originally center

\renewenvironment{abstract}
 {{%
    \setlength{\leftmargin}{0mm}
    \setlength{\rightmargin}{\leftmargin}%
  }%
  \relax}
 {\endlist}

\makeatletter
\def\@maketitle{%
  \newpage
%  \null
%  \vskip 2em%
%  \begin{center}%
  \let \footnote \thanks
    {\fontsize{18}{20}\selectfont\raggedright  \setlength{\parindent}{0pt} \@title \par}%
}
%\fi
\makeatother




\setcounter{secnumdepth}{3}



\title{Título\\
Subtítulo\\
Subtítulo  }



\author{\Large Cinthia Amalia Vandepool Candelario\vspace{0.05in} \newline\normalsize\emph{Estudiante de Geografia, Universidad Autónoma de Santo Domingo (UASD)}  }


\date{}

\usepackage{titlesec}

\titleformat*{\section}{\normalsize\bfseries}
\titleformat*{\subsection}{\normalsize\itshape}
\titleformat*{\subsubsection}{\normalsize\itshape}
\titleformat*{\paragraph}{\normalsize\itshape}
\titleformat*{\subparagraph}{\normalsize\itshape}

\titlespacing{\section}
{0pt}{36pt}{0pt}
\titlespacing{\subsection}
{0pt}{36pt}{0pt}
\titlespacing{\subsubsection}
{0pt}{36pt}{0pt}





\newtheorem{hypothesis}{Hypothesis}
\usepackage{setspace}

\makeatletter
\@ifpackageloaded{hyperref}{}{%
\ifxetex
  \PassOptionsToPackage{hyphens}{url}\usepackage[setpagesize=false, % page size defined by xetex
              unicode=false, % unicode breaks when used with xetex
              xetex]{hyperref}
\else
  \PassOptionsToPackage{hyphens}{url}\usepackage[unicode=true]{hyperref}
\fi
}

\@ifpackageloaded{color}{
    \PassOptionsToPackage{usenames,dvipsnames}{color}
}{%
    \usepackage[usenames,dvipsnames]{color}
}
\makeatother
\hypersetup{breaklinks=true,
            bookmarks=true,
            pdfauthor={Cinthia Amalia Vandepool Candelario (Estudiante de Geografia, Universidad Autónoma de Santo Domingo (UASD))},
             pdfkeywords = {morfometría fluvial, modelo digital de elevación, red de drenaje},  
            pdftitle={Título\\
Subtítulo\\
Subtítulo},
            colorlinks=true,
            citecolor=blue,
            urlcolor=blue,
            linkcolor=magenta,
            pdfborder={0 0 0}}
\urlstyle{same}  % don't use monospace font for urls

% set default figure placement to htbp
\makeatletter
\def\fps@figure{htbp}
\makeatother

\usepackage{pdflscape} \newcommand{\blandscape}{\begin{landscape}}
\newcommand{\elandscape}{\end{landscape}}


% add tightlist ----------
\providecommand{\tightlist}{%
\setlength{\itemsep}{0pt}\setlength{\parskip}{0pt}}

\begin{document}
	
% \pagenumbering{arabic}% resets `page` counter to 1 
%
% \maketitle

{% \usefont{T1}{pnc}{m}{n}
\setlength{\parindent}{0pt}
\thispagestyle{plain}
{\fontsize{18}{20}\selectfont\raggedright 
\maketitle  % title \par  

}

{
   \vskip 13.5pt\relax \normalsize\fontsize{11}{12} 
\textbf{\authorfont Cinthia Amalia Vandepool Candelario} \hskip 15pt \emph{\small Estudiante de Geografia, Universidad Autónoma de Santo Domingo (UASD)}   

}

}








\begin{abstract}

    \hbox{\vrule height .2pt width 39.14pc}

    \vskip 8.5pt % \small 

\noindent Resumen del manuscrito


\vskip 8.5pt \noindent \emph{Keywords}: morfometría fluvial, modelo digital de elevación, red de drenaje \par

    \hbox{\vrule height .2pt width 39.14pc}



\end{abstract}


\vskip 6.5pt


\noindent  \section{Introducción}\label{introducciuxf3n}

La morfometría fluvial se encarga de analizar los parámetros
geomorfológicos de una cuenca hidrográfica, tales como la red de
drenaje, la pendiente, la forma, el orden de la red y demas aspectos
fisicos. Por cuenca hidrográfica entendemos ese sistema o unidad
geográfica e hidrológica, formada por un rio principal y todo el
territorio entre el origen del rio y su desembocadura, en este espacio
interactúan factores bióticos y abióticos.

La cuenca hidrográfica a analizar en esta investigación es la Subcuenca
Caña perteneciente a la Cuenca del rio Macasia, ubicada en el extremo
suroeste de la republica dominicana, dicho análisis se realizara
basandonos en datos preexistentes a partir de un \emph{modelo digital de
elevación (DEM)}, el cual es un modelo simbólico, de estructura numérica
y digital que pretende representar la distribución espacial de la
elevación del terreno, siendo la altura una variable escalar que se
distribuye en un espacio bi-dimensional (Burgos \& Salcedo (2014)).

\section{Metodología}\label{metodologuxeda}

Para la elaboracion de esta investigación se emplearon metodos de
analisis morfometrico a partir de un DEM de la cuenca de interes,
inicialmente cargue una serie de paquetes de Grass en R para adecuar el
entorno de R para los ejectar los codigos necesarios.

En primer lugar, importe a R, como SpatialGridDataFrame, un DEM alojado
en la base de datos de GRASS GIS, estableci la ruta y lo converti en un
objeto raster usando el paquete raster de R, partiendo del complemento
\emph{r.watershed}, el cual genera un conjunto de mapas que indican:
\emph{la acumulación de flujo, la dirección del drenaje, la ubicación de
los arroyos y las cuencas hidrográficas} (Team (2003b)), y del modelo
digital de elevaciones (DEM) genere capas y calcule los parámetros
hidrográficos de la cuenca del rio caña y sus redes de drenaje, ademas,
seguido a esto importe un conjunto de capas ráster de GRASS GIS a R,
como el mapa de red de drenaje y el de cuencas visualizandolas con
\emph{leaflet}.

Utilizando el complemento de GRASS GIS \emph{r.water.outlet} y
apoyandome en los paquetes \emph{mapview} y \emph{leaflet} extraje una
cuenca de drenaje a partir de un mapa de dirección de flujo y la
coordenada de desembocadura de la cuenca cana (-71.62524,18.94026).

El complemento \emph{r.water.outlet} se encarga de genera una cuenca
hidrográfica a partir de un mapa de dirección de drenaje y un conjunto
de coordenadas que representan el punto de salida de la cuenca. El mapa
de dirección de drenaje de entrada indica el ``aspecto'' de cada celda
(Team (2003a)).

Posteriormente establecí una máscara usando el límite de la cuenca caña
para luego realizar la extraccion, partir del DEM, de la red de drenaje
utilizando el complemento de GRASS GIS \emph{r.stream.extract} desde R.
Tras esto, utilice el complemento \emph{r.stream} para generar un mapa
de dirección de flujo, \emph{r.stream.order} para un mapa de orden de
red según varios métodos, entre ellos Strahler y Horton, a partir de
\emph{r.stream.basins} un mapa de cuencas según órdenes de red y con
\emph{r.stream.stats} genere las estadísticas de red resumidas por
órdenes y expandidas, incluyendo la razón de bifurcación.

\section{Resultados}\label{resultados}

\ldots

\section{Discusión}\label{discusiuxf3n}

\section{Agradecimientos}\label{agradecimientos}

\section{Información de soporte}\label{informaciuxf3n-de-soporte}

\ldots

\section{\texorpdfstring{\emph{Script}
reproducible}{Script reproducible}}\label{script-reproducible}

\ldots

\section*{Referencias}\label{referencias}
\addcontentsline{toc}{section}{Referencias}

\hypertarget{refs}{}
\hypertarget{ref-burgos2014modelos}{}
Burgos, V. H., \& Salcedo, A. P. (2014). Modelos digitales de elevación:
Tendencias, correcciones hidrológicas y nuevas fuentes de información.
\emph{Encuentro de Investigadores En Formación En Recursos Hídricos (2,
2014, Ezeiza, Buenos Aires, Argentina). Disponible En: Http://Www. Ina.
Gov. Ar/Ifrh-2014/Eje1/1.11. Pdf. Consultado}, \emph{1}(10), 2015.

\hypertarget{ref-addonrwateroutlet}{}
Team, G. D. (2003a). R.water.outlet - creates watershed basins from a
drainage direction map. Retrieved April 2, 2021, from
\url{https://grass.osgeo.org/grass78/manuals/r.water.outlet.html}

\hypertarget{ref-addonrwater}{}
Team, G. D. (2003b). R.watershed - calculates hydrological parameters
and rusle factors. Retrieved April 2, 2021, from
\url{https://grass.osgeo.org/grass76/manuals/r.watershed.html}




\newpage
\singlespacing 
\end{document}
